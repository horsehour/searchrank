\chapter{社交网络中的好友推荐}
%\author{蒋春恒}
\date{2012-06-19}
\section{摘要}
Web2.0的蓬勃发展,表明全球互联网SNS时代已经来临。各种类型的SNS网站不断涌现,如国外的Facebook \footnote{\url{http://www.facebook.com/},社交网络服务网站,2004年上线,2012年5月18日正式在美国纳斯达克证券交易所上市},Twitter\footnote{\url{http://www.twitter.com/},社交网络和微博服务网站,2006年上线} ,LinkedIn\footnote{\url{http://www.linkedin.com/},面向商户的社交网络服务网站, 成立于2002年12月,并于2003年启动} ,Flickr\footnote{\url{http://www.flickr.com/},提供免费及付费照片储存、分享方案之线上服务,也提供网络社群服务的平台,被认为是Web 2.0最佳应用} ,国内的人人网\footnote{\url{http://www.renren.com/},原为校内网,2009年更名,是中国最大社交网站} 、 新浪微博\footnote{\url{http://www.weibo.com/},由新浪推出的类似于Twitter的微博网站} 。在这些网站上,用户可以通过添加好友来扩展自己的人脉。然而,如何选择好友并不容易,好友推荐系统即是解决这一问题的途径。本文主要从好友关系的度量和好友推荐模型两个方面回顾好友推荐算法的进展,并提出好友推荐所面临的问题。

一个优秀的推荐系统不仅能够提升用户体验和粘着性,而且从商业角度来看,有利于广告的精确投放。因此,各个社交网站都在努力提升各自推荐系统的性能和效果,其中,好友推荐是推荐系统中的关键一环。

\section{好友推荐}
从1997年上线的SixDegrees\footnote{\url{http://sixdegrees.com/},是第一个社交服务网站,1997年上线,名称源自“六度分离”}到当前有效用户量超过5亿的Facebook,社交网络经历了天翻地覆的变化,目前Facebook 已经发展成为一个为用户提供生活、社会、文化、情感、娱乐、文学、经济、教育、科技、体育等综合信息的社交网站。社交网络为互联网用户提供了一个真实的网络生活平台,已然成为人们生活中必不可少的一部分。用户可以使用它获得即时资讯,结交志同道合的朋友,自由表达自己的思想。

\subsection{推荐系统}
在电子商务的虚拟环境下,商家所提供的商品种类和数量非常多,用户不可能通过一个小小的计算机屏幕一眼就知道所有的商品,用户也不可能像在物理环境下那样检查挑选商品。因此,需要商家提供一些智能化的选购指导,根据用户的兴趣爱好推荐用户可能感兴趣或是满意的商品,使用户能够很方便地得到自己所需要得到的商品。而且,从现实经验来看,用户的需求经常是不明确的、模糊的,可能会对某类商品有着潜在的需求,但并不清楚什么商品能满足自己的模糊需求。这时,如果商家能够把满足用户模糊需求的商品推荐给用户,就可以把用户的潜在需求转化为现实的需求,从而提高产品的销售量。

在这种背景下,推荐系统应运而生,它是根据用户的特征,比如兴趣爱好,推荐满足用户要求的对象,也称个性化推荐系统。实际中应用最多的,是在网上购物(如Amazon\footnote{\url{http://www.amazon.com/}},淘宝\footnote{\url{http://www.taobao.com/}})环境下的、以商品为推荐对象的个性化推荐系统,它为用户推荐符合兴趣爱好的商品,如书籍,音像等。

目前,推荐系统已经运用到多个行业中,推荐对象包括书籍、音像、网页、文章、新闻等,各种类型的网站也争相开发出各自的推荐系统,社交网站也不例外。

一般地,推荐系统可以分成两种基本类型\cite{liu2009personalizedrs},包括基于内容的推荐和协同过滤技术。

基于内容的推荐主要采用自然语言处理、人工智能、概率统计和机器学习等技术,根据对象的相关属性为对象建立特征表示,并使用对象的特征表示衡量对象间的相似度,按照相似度大小为用户推荐其他对象。基于内容的推荐直观且易于实现,因此也是最早被采用的方法。基于内容的推荐同样存在着一些缺陷:(1)由于基于内容的推荐利用特征属性表示对象,因此需要预先抽取对象特征,但是对于某些对象,特征抽取构成难题;(2)基于内容的推荐对用户历史行为太过依赖,因此无法挖掘出用户的潜在兴趣,也限制了推荐的多样性;(3)对于新用户,并没有太多历史行为可资使用,推荐无法进行或效果不佳。

协同过滤技术是目前最流行的推荐方法,它考虑到用户之间的协同作用对推荐的影响,根据单个用户的历史行为(购买、选择或者评价),并结合其他用户做出的相似决定,建立一种有效的预测模型,为用户提供可能感兴趣的对象。协同过滤没有特征抽取的问题,也克服了推荐范围过窄的缺陷,但是与基于内容的推荐同样经受“冷启动”的困扰,即对于新用户或新的对象,由于可供挖掘的数据不足,推荐系统无法有效运行。在实际运用中,常常将两种方法融合,构成混合模型,融合方法不同,则混合模型也不同。

目前,社交网站一般都会提供基本的信息发布功能,如发布状态、分享评价音频图片等等,允许用户修改个人头像、基本档案(年龄、性别、毕业院校等),提供基本的交友交流平台(如申请好友、订阅好友、与好友互动等),在方便用户获取信息的同时,向用户推荐潜在好友,以扩展用户的人脉网络。

社交网络中好友推荐是增强用户黏着、提升用户体验的一种有效的方式。在了解基本的推荐理论以后,就需要根据社交网络平台所记录的用户基本信息及在线行为,建立一个合理的预测模型,为用户推荐潜在的好友。

\subsection{图和链接分析}
无论是互联网,还是真实的人际关系,甚至全球各国之间的贸易往来都构成一个图。对于互联网,每一个网页都可以看作一个结点,而网页之间的超链接则构成有向边,因此可以认为互联网就是一个有向图。在全球化的今天,如果把国家看作结点,国家之间通过进出口贸易进行互联,于是就形成了一个纵横交错的贸易网络,从本质上来说,它是一个有向图。很多问题,运用图的思想便可迎刃而解。

社交网络图谱中,用户是结点,用户之间的关系(好友,同学,家人,同事等)形成图中的边,至于是有向图还是无向图,则因问题不同而有所区别。

传统的信息检索技术主要根据关键词匹配,从海量的文档数据库中寻找与关键词最相关的结果。然而,单纯基于文本的检索技术在实际应用中,如使用搜索引擎检索文档时效果并不理想。1998年,Sergey Brin和 Lawrence Page提出的PageRank \cite{brin1998anatomy}算法极大地改善了搜索引擎的搜索效果,并成为Google的核心算法之一。PageRank算法不仅考虑链接的数量,同时还根据网页/网站本身的质量差异为链接赋予不同的权重。同年Jon Kleinberg提出HITS算法\cite{kleinberg1999authoritative}。

\subsubsection{PageRank}
Google如今已经成为世界上最好的搜索引擎之一。它取得巨大成功的关键之处在于利用链接特征衡量网页的威信度(Authority),并据此对网页进行排序,在用户提交检索关键词后,以列表的形式将网页按照威信度大小展示给用户。

PageRank的基本思想十分简单,它将网页中的超链接看作是对链接所指页面的投票,并且页面所含链接数目越少,则其投票的权重越大,于是对于某个网页,它所得到的票数越大,而且每张票的权重越大,那么其威信度也就越高。PageRank算法是一个随机冲浪模型,假设冲浪者浏览网页是随机游走,可以建立如下形式的模型:
\begin{equation}
PR(U) = \frac{1-d}{N} + d\sum_{V\rightarrow U}{\frac{PR(V)}{N(V)}}
\end{equation}
其中,网页U、V的PageRank值表示成PR(U)、PR(V);N(V)表示网页V中包含的超链接数目,即出链;$V\rightarrow U$表示从V到U存在超链接;参数d表示阻滞因子,是用户通过超链接浏览网页的时间比例(其余时间则是使用地址栏随机输入网址浏览网页),$0\le d \le 1$;N表示整个网络的大小。

\subsubsection{HITS}
Jon Kleinberg提出的HITS算法基于链接分析,引入两种评价指标——威信度(Authority Score)和辐射度(Hub Score)来衡量网页的权重。Kleinberg认为包含诸多优质资源链接的网页,对于网页冲浪者同样重要,比如典型的门户网站Yahoo!\footnote{\url{http://www.yahoo.com/}}。

HITS算法首先根据用户提交的检索词,根据传统的文本检索模型,从互联网上抽取与检索词相关性最大的一组网页构成根集,对于根集内的每个网页,通过引入该网页所指页面以及指向该页面的全部网页,扩展为基本集。基本集包含了与检索词主题相关而且威信度较高的多数网页,因此又被称作主题型子图。直观地看,在子图上,高威信度的网页应该有很多高辐射度的网页引用它,而高辐射度的网页则通常会引用许多高威信度的网页。这种相互增强的关系可以由以下两式来表示:
\begin{equation}\label{eq:hitsalgorithm}
    \begin{array}{cc}
        Aut(U)=\sum\limits_{V\rightarrow U,V \in B(Q)}{Hub(V)}, & Hub(U)=\sum\limits_{U\rightarrow V,V \in B(Q)}{Aut(V)}
    \end{array}
\end{equation}
其中,Aut(*)、Hub(*)分别表示网页的威信度和辐射度;B(Q)表示根据查询词Q扩展得到的基本集。
\section{好友关系的度量}
在数据挖掘(分析)过程,我们经常需要研究对象间(用户、网页、音乐、图书等)的差异大小,以此来衡量对象间的相似性和相关性,常见的如数据挖掘中的分类和聚类算法,数据分析中的相关性分析。

在实际应用中,衡量对象间差异大小的度量方式有很多种,本节将列出几种比较常见的度量。

为方便表述,假设$X=(x_1,x_2,\ldots,x_n)\in \mathbb{R}^n$,$Y=(y_1,y_2,\ldots,y_n)\in \mathbb{R}^n$,$Dist(X,Y)$和$Sim(X,Y)$分别表示两向量的距离和相似度。

\subsection{距离度量}
距离度量衡量对象间的差异程度,距离越远则对象间的差异越大。

\subsubsection{闵可夫斯基距离(Minkowski Distance)}
闵可夫斯基距离是对多个距离度量公式的概括性的表述,公式如下:
\begin{equation}\label{eq:minkowski}
Dist(X,Y)=\sqrt[p]{\sum_{i=0}^{n}{(x_i-y_i)^p}}
\end{equation}

\subsubsection{欧几里得距离(Euclidean Distance)}
欧氏距离是最常见的距离度量,衡量的是多维空间中点与点之间的绝对距离,公式如下:
\begin{equation}\label{eq:euclidean}
Dist(X,Y)=\sqrt{\sum_{i=0}^{n}{(x_i-y_i)^2}}
\end{equation}
对于闵可夫斯基距离,当$p=2$时即得到欧几里得距离。

\subsubsection{曼哈顿距离(Manhattan Distance)}
曼哈顿距离来源于城市区块距离,它是将多个维度上的距离求和后的结果,当闵可夫斯基距离中的$p=1$时即得到如下公式:
\begin{equation}\label{eq:manhattan}
Dist(X,Y)=\sum_{i=0}^{n}{|x_i-y_i|}
\end{equation}

\subsubsection{切比雪夫距离(Chebyshev Distance)}
切比雪夫距离起源于国际象棋中国王的走法。切比雪夫距离是当$p$趋向于无穷时的闵可夫斯基距离:
\begin{equation}\label{eq:chebyshev}
Dist(X,Y)=\lim_{p \rightarrow \infty}{\sqrt[p]{\sum_{i=0}^{n}{(x_i-y_i)^p}}}=\max_{i}{|x_i-y_i|}
\end{equation}

\subsection{相似度度量}
相似度度量,即计算对象间的相似程度,与距离度量相反,相似度度量值越小,则说明对象间相似度越小,差异越大。

\subsubsection{余弦相似度(Cosine Similarity)}
余弦相似度根据向量空间中两个向量夹角的余弦值作为衡量两个对象间差异的大小。公式如下:
\begin{equation}\label{eq:cosinesim}
Sim(X,Y)=\frac{X^TY}{||X||||Y||}
\end{equation}

\subsubsection{皮尔逊相关系数(Pearson Correlation Coefficient)}
皮尔逊相关系数即简单相关系数,描述了两个变量间的关联程度。公式如下:
\begin{equation}\label{eq:pearsoncor}
R(X,Y)=\frac{Cov(X,Y)}{\sqrt{Cov(X,X)Cov(Y,Y)}}=\frac{nX^TY-(X^TI)(Y^TI)}{\sqrt{(nX^TX-(X^TI)^2)(nY^TY-(Y^TI)^2)}}
\end{equation}
根据定义,皮尔逊相关系数$-1 \le R(X,Y) \le 1$,通过$|R(X,Y)|$的大小可以判定相关程度,$|R(X,Y)|$越大,X和Y的相关程度越大。当$R(X,Y)=1$时,表示X和Y完全正相关;当$R(X,Y)=-1$表示完全负相关;当$R(X,Y)=0$表示不相关。

\subsubsection{Jaccard相似系数(Jaccard Coefficient)}
Jaccard相似系数,利用集合的概念衡量对象间的相似程度,它只关心具有相同特征的比例。公式如下:
\begin{equation}\label{eq:jaccardsim}
R(X,Y)=\frac{X\cap Y}{X\cup Y}
\end{equation}

\subsection{用户相似度度量}
在社交网络中,用户之间的好友关系既可以是二元的,如是否为朋友,也可以是多元的,如血亲、密友、点头之交、同学、同事等。两种表示方式都是对用户群的一种划分,都可以归结为数据挖掘中的分类或聚类问题。

本文主要关注用户之间的好友关系,并希望对好友关系精确量化,以增加模型构建的灵活性。

为简便起见,以下均假设U和V是同一个社交网络平台下的任意两个不同用户,设$Rel(U,V)$表示两个用户之间的关系。

\subsubsection{二元关系}
\begin{equation}\label{eq:bi-relation}
Rel(U,V) = \left\{
    \begin{array}{l}
        1, \text{如果U和V是好友}\\
        0, \text{否则}
    \end{array}
    \right.
\end{equation}

\subsubsection{离散型多元关系}
\begin{equation}\label{eq:multidis-relation}
Rel(U,V)=i,\text{如果U和V的关系类型是}T_i,i=1,2,\dots
\end{equation}
这里,U和V的关系使用关系类型标号表示,如果只考虑是否为好友关系的话,即可转化为二元关系。

好友关系的确立,无论是单向的还是相互的,都必须保证二者之间存在“链接”,即通过发送交友请求,并为对方接受所确定下来的一种关系。

\subsubsection{连续数值型多元关系}
社交网络中记录了全部用户的基本档案信息(年龄、性别、毕业院校等)以及他们各自的交互行为,如评价、分享等,这些数据在衡量任意两个用户,如U和V之间的相似性时都有或多或少的作用。一个优秀的推荐系统能够提供精确的推荐,需要首先从海量的异质数据中寻找真实反映用户特征的主要属性,这一环节我们称为“特征选取”,我们留待后文再做介绍。

从用户的基本属性中抽取出d个主要特征,构成一个d维特征空间$\Theta=(f_1,f_2,\ldots,f_d )^T$,则对于用户U和V可以使用两个d维向量$\Theta_U$,$\Theta_V$分别予以表示,其中
\[\Theta_U=(f_{U1},f_{U2},\ldots,f_{Ud})\]
\[\Theta_V=(f_{V1},f_{V2},\ldots,f_{Vd})\]

特征空间$\Theta$实际上是用户的一种归约化的表现,是用户行为特征的近似,这也是数据挖掘中经常使用的一项技术。

现在,我们只要选择一种合适的度量(相似度或距离),即可确定用户U和V之间的相似度或者二者的距离。此外,还存在其他形式的度量,如有返回的随机游走模型
\cite{tong2008random,fujiwara2012fast}, 、EdgeRank\footnote{\href{http://techcrunch.com/2010/04/22/facebook-edgerank/}{\textit{EdgeRank: The Secret Sauce That Makes Facebook's News Feed Tick}}}、UserSimilarity\cite{gretarsson2010smallworlds}、Interactions Rank\cite{roth2010suggesting} , 等。

\subsubsection{Random Walk with Restart}
Sun等人在2005年提出的有返回的随机游走(Random Walk with Restart,记作RWR)模型~\cite{sun2005neighborhood}可以有效的计算网络图中的任意两个结点之间的相近程度。RWR无论是在应用还是理论方面都引起了研究人员的兴趣,并被成功运用到自动图像命名(Auto-Image Captioning)、推荐系统(Recommender System)和链接预测(Link Predict)\cite{fujiwara2012fast}。

有返回的机游走模型主要用于度量图中任意两结点之间的结构相关性。所谓有返回的随机游走是指,用户从图中的某个结点出发,沿着图中的边顺次逐步走下去。在每一步中,用户都存在返回起始结点的可能性,其概率为$c$。假设起始结点是j,根据有返回的随机游走假设建立如下模型:
\begin{equation}\label{eq:rwr}
R=(1-c)AR + ce_j
\end{equation}
其中,R是一个多维向量,向量的每个元素表示对应结点被访问的概率,维数等于图中结点的总数;c表示返回概率;A是一个列随机矩阵,同PageRank模型中的邻接矩阵一致;$e_j$表示第j个元素为1的单位矩阵。

可以证明,使用迭代算法可以得到各个结点的稳态概率值。初始结点的选择反映了用户的个人偏好,而其他结点的稳态概率值是平均意义上的访问频率,因此可以看作是相对于起始结点(比如结点j)的相近程度。

\subsubsection{EdgeRank}
Facebook于2006年9月推出News Feed,并改变了Facebook的运作方式,但是出于商业秘密,Facebook官方关于News Feed的信息很少,Jason Kincaid估计News Feed的成功很大程度上归结于EdgeRank。用户的News Feed的每一条信息都是一个对象,如果其他用户与之交互,则立即创建一条“边(Edge)”。每条边包含三个重要的组成部分:(1)用户之间的亲密分值(Affinity Score):如果你向某个用户发送多条信息,并经常造访其主页,则你对该用户的亲密分值就越高;(2)“边”的类型:不同的“边”,权重也不同,评论相对于“Like”分值更高;(3)时间:“边”越旧,则权值越低。

综合考虑以上几种分值,就可以获得“边”的排名——EdgeRank,分值越高,则对应的信息更可能出现在用户的News Feed中。从决定EdgeRank的三个因素来看,EdgeRank是一个不错的用以衡量用户之间亲密程度的度量,EdgeRank 越大,则用户之间越亲密。

\subsubsection{SmallWorlds}
可视化数据挖掘目前已经成为数据挖掘领域中的一个热门研究方向,它更强调用户同图形界面的交互,对数据的表现相对传统数据挖掘技术更加形象,易于理解。SmallWorlds使用Facebook提供的API获取用户数据,引入一款可视化图形交互工具SmallWorlds,运用自动协同过滤(ACF)技术,组合用户的交互信息和直接邻居的协同信息,建立预测模型,为用户推荐未知项目。

推荐系统的关键一环是确定用户的邻居,SmallWorlds使用的方案是,针对用户的评价数据,计算用户间的相关程度(如夹角余弦或皮尔逊相关系数),根据相关性确定直接邻居。

假设$IW$表示项目权重(Item Weight),$TWCI$表示用户的公共项目的总权重(Total Weight of Common Items),$TWI$指用户档案所含项目的总权重(Total Weight of Items),$UW$指用户权重(User Weight),$Likes(U,i)$表示一种二元关系,用户U喜欢项目i则为1,否则为0。

系统对各个变量做如下定义:
\begin{enumerate}[(1)]
\item 首先为每个项目以及激活用户的预设权重1,然后根据Drag distance调整,最大值不超过5,即$1 \le IW(i) \le 5$,$1 \le UW(U) \le 5$,其中i为目标用户档案中的项目,U为目标用户。

\item 然后,分别计算目标用户U和好友V的公共项目的总权重和用户U的项目总权重:
\[TWCI(U,V)=\sum_{i \in I}{(Likes(U,i)*Likes(V,i)*IW(i))}\]
\[TWI(U)=\sum_{i \in I}{(Likes(U,i)*IW(i))}\]
根据定义,$TWCI(U,V)$反映U和V存在共同偏好的程度。

\item 最后,计算用户之间的相似程度:
\[Sim(U,V)=\frac{UW(U)*TWCI(U,V)}{\sqrt{TWI(U)*TWI(V)}}\]

\end{enumerate}

\subsubsection{Interactions Rank}
社交网络中好友关系常常是非对称性的,Roth et al.提出的Interactions Rank算法\cite{roth2010suggesting} 通过判断用户之间的交互类型,赋予不同的权值,解决了这个问题。Interactions Rank主要考虑三个方面的因素:交互频率、交互的时效性和交互的方向,它使用用户的一组交互数据,定义如下形式的数学模型:
\begin{equation}\label{eq:interactrank}
IR=\omega_{out} \sum_{i \in I_{out}}{0.5^{\frac{t_0-t_i}{\lambda}}+\sum_{i \in I_{in}}{0.5^{\frac{t_0-t_i}{\lambda}}}}
\end{equation}

其中,$I_{out}$表示由用户向好友圈发出的交互行为,$I_{in}$表示用户的好友圈向其发出的交互行为,$t_0$ 表示当前时间,$t_i$则表示交互行为发生的时间点。参数$\lambda$用于调整交互行为的时新性对Interaction Rank 分值的影响大小,$\omega_{out}$衡量由用户主动发起的交互行为相对被动交互行为的重要程度,两个参数都可以动态地调整。此外,交互行为的时新性对Interactions Rank的影响是呈指数型衰减的。

\section{好友推荐模型}
假设某社交网站有$n$个用户,网站的网络图为$G=(U,E)$。其中,$U=\{u_i\}_{i=1}^{n}$表示网站的所有用户集合,$E$表示所有用户关系的集合。令$C=(c_{ij})$表示网络图$G$的邻接矩阵,其中
\[
    c_{ij}=\left\{
        \begin{array}{l}
        1, \text{如果用户$j$是用户$i$的好友}\\
        0, \text{否则}
        \end{array}
    \right.
\]

由于在大多数社交网站,好友关系不是相互的(用户$j$是$i$的好友并不能保证$i$也是$j$的好友),因此C不一定是对称矩阵。

传统好友推荐引擎的基本工作流程可以概括为以下两个步骤:
\begin{enumerate}[(1)]
\item 计算网络图$G$上任意两个用户之间的相似程度。
\item 根据预先给定的相似度阈值或近邻数目,确定用户的近邻,作为系统向目标用户的待推荐的好友集合,也称潜在好友集合。
\end{enumerate}

我们可以定义如下形式的潜在好友集合:
\begin{equation}\label{eq:friendrecsy}
S_v=\{u\in U|Sim(u,v) \ge \gamma \text{或} u \in \tau(v,k)\}
\end{equation}
其中,参数$\gamma$表示预设的相似度下限,$k$表示预设的近邻数目,比如20,而$\tau(u,k)$则表示与$u$最亲密的$k$个好友集合,用以衡量好友亲密程度的可以是这里的相似度,也可以是别的度量。

决定推荐精度的主要环节在于确定用户之间的相似程度。实际上,系统只有把握住用户的关键属性/特征,所确定的用户间相似度度量才有意义。因此,需要对社交网络中的用户行为有个比较准确的认识。

根据观察可知,社交网络中的用户行为可以从三个方面描述。(1)创建用户档案:通常来说,注册用户会自动创建个人档案。它是个人信息的基本概况,主要包括年龄、性别、毕业院校、爱好等。(2)生成内容:在社交网络中,由用户行为所产生的一切内容都属于内容创建的范畴,比如创作的日志、分享的图片视频、评论和评级等等。(3)建立关系:由于用户之间形成交互信息,则说明用户之间已经建立了某种关系,如相互访问主页、评论留言等,而且这种关系有时是有方向的。

根据用户行为所确立的关系,可以建立一种简单的、直观的模型,即根据用户间共同好友的数目为标准,共同好友越多,则作为潜在好友向目标用户推荐。

然而,由于地域、种族、信仰等因素,仅仅依靠共同好友的数目来推荐并不可靠。因此需要考虑到用户档案信息,从中抽取反映用户特点的属性。Bhattacharyya等人提出基于关键词分析用户档案的语义关系以确定用户之间的相似性\cite{bhattacharyya2011analysis}。Bian等使用自动协同过滤系统MatchMaker,以一种十分巧妙的方式确定用户之间的相近程度\cite{bian2012matchmaker}。它首先根据Facebook用户的在线档案和电视角色的档案信息建立配对,再结合电视节目中各个角色的关联信息形成对Facebook用户的反馈。

目前,移动设备已经成为大众化的交流媒介,手机的使用产生了海量的数据,如实时GPS位置、移动轨迹等,均反映了用户的日常活动和真实生活的社交互动。地理信息在好友推荐系统中逐渐被采纳,如微软亚洲研究院的GeoLife项目使用GPS数据,掀起了研究GPS数据智能化处理的高潮,并分别基于单人的轨迹数据、包含语义的轨迹数据向用户推荐好友或者位置信息\cite{li2008mining,xiao2010finding,zheng2010geolife,zheng2011recommending}。

2011年,Yu等人提出一种基于地理的好友搜寻框架,也是使用GPS轨迹数据,首先从原始的GPS轨迹数据中挖掘出潜在的几种模式,如FL-模式、FT-模式、FLT-模式、FTT- 模式,然后构建基于模式的异质信息网络,在传统社交网络的基础上引入用户的GPS位置信息,形成一个包含模式到用户和用户到模式的综合信息网络,并使用有返回的随机游走模型为目标用户推荐好友。

Yao等人提出基于上下文的好友推荐系统\cite{yao2011context},同时考虑地理和可视化线索,并开发出对应的基于用户的地理相似度(Geo Similarity)和可视化相似度(Visual Similarity)。

Google是当前世界最顶尖的搜索引擎公司,曾经是互联网用户进入互联网的主要入口,这种作用正在逐渐被Facebook等社交网络平台侵蚀,尽管曾经发起过在社交网络领域的多次攻势\footnote{\url{http://www.wired.com/business/2011/06/inside-google-plus-social/}},如2004年发布的Orkut、2007 年创建名为Open Social的社交网络应用开放标准、2009 年在Google I/O 大会上介绍的社会化的交流系统Wave、同年又试图以Buzz攻陷社交世界,但均以失败告终。2011年又推出Google+,其命运如何,我们将拭目以待。

Google科学家最近推出了基于用户交互的社交网络图谱算法Interactions Rank。它定义用户与好友圈子之间的交互类别,并对不同的交互行为进行打分,找出与用户最亲密的好友圈子。目前,Interactions Rank模型已被Gmail用来为邮箱用户推荐可能的收件人。

\section{好友推荐的挑战}
好友推荐面临几下几种挑战:
\begin{enumerate}[(1)]
\item 用户的“特征”抽取。对于标准化的用户信息,抽取时较为方面,然而,由于用户档案信息中包含众多隐含的,不明确的内容,而且维数较高,难以理解,甚至包含许多噪声,因此,需要明确抽取的特征范围,并能够识别和去除干扰噪声。一般而言,用户的特征可以分为“显式维”(如年龄,学历,关注人数,粉丝数目,评论量,转发数量,收藏标签和数目,话题分类等等),以及隐藏用户心理信息的“隐式维”
(如职业,兴趣,性格等)。

\item 自动识别用户“特征”及学习能力。若实现推荐系统的智能自动识别,需要配置相应的语义分析模块,仔细并准确的分析语义关系,抽取出有价值的信息,对用户向量化表示。由于不同的社交网络甚至同一社交网络的结点由于连带作用,各个结点相互影响,从而对整个网络的结构产生影响,从而需要一个具备学习能力,能够自动识别出用户重要“特征”的推荐系统,既减少认为操作的失误,同时又节约的大量的人工投入。

\item 信息质量检测体系。目前的推荐系统对于网络中的全部信息几乎是一视同仁,并未在预处理阶段作出初步判定,或者对那些干扰用户正常使用,信息冗余的内容进行过滤,从而一方面造成多余无用的信息抽取工作,同时又是引入噪声数据的根源。因此,需要一个独立的检测体系,在预处理阶段,对用户信息进行基本的检测,从而节约资源投入,并提高推荐的洁净度。

\item 用户体验。任何公司都在宣扬以用户为中心,或者用户至上的理念,然而实际操作中,由于需要兼顾盈利等经济压力,从而在设计时,就更多的往利润方面倾斜,从而导致用户体验本可以提高,但仍未提高的局面。其实,任何设计都应该以用户体验作为衡量产品成功与否的标准,毕竟有了好的用户体验,再辅以良好的推介和内容包装,自然用户就会为SNS带来可观的收入。
\end{enumerate}
