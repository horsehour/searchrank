\chapter{国际会议与期刊}
计算机科学方面的论文,最大的特点:极度重视会议,而期刊则通常只用来做re-publication。大部分期刊文章都是会议论文的扩展版,首发就在期刊上的相对较少。因此,计算机期刊的影响因子都比较低,顶级刊物也只有1$\sim$2左右。会议论文被引用的次数要多于期刊论文。

\section{国际会议}
\begin{itemize}
  \item ICML - International Conference on Machine Learning
  \item ECML - European Conference on Machine Learning
  \item NIPS - Neural Information Processing Systems
  \item COLT - Conference on Learning Theory
  \item ICDM -  International Conference on Data Mining (IEEE)
  \item WWW -  International World Wide Web Conference (IEEE)
  \item CIKM - Conference on Information and Knowledge Management (ACM)
  \item WSDM - International Conference on Web Search and Data Mining (ACM)
  \item CVPR - Computer Vision and Pattern Recognition
  \item SIGIR - Special Interests Group on Information Retrieval (ACM)
  \item SIGKDD - Special Interests Group on Knowledge Discovery in Databases (ACM)
  \item RecSys - Conference on Recommender Systems (ACM)
  \item IJCAI - International Joint Conference on Artificial Intelligence
  \item AAAI - Association for the Advancement of Artificial Intelligence
  \item ICIP - International Conference on Image Processing
  \item ICCV - International Conference on computer vision (IEEE)
\end{itemize}
\section{期刊}
\begin{itemize}
  \item Journal of Machine Learning Research
  \item Machine Learning
  \item IEEE Transaction on Pattern Analysis and Machine Intelligence
  \item Neural Computing
  \item IEEE Transaction on Neural Network
  \item IEEE Transaction on Knowledge and Data Engineering
\end{itemize}

\section{会议}
\begin{enumerate}[(1)]
\item ACM SIGIR: \href{http://www.sigir.org/}{Special Interest Group on Information Retrieval}.

从2007年开始,L2R成为会议的一个独立主题。\href{http://sigir2013.ie/}{The 36th ACM SIGIR Annual Conference} will be held in Dublin Ireland, on 28 July -1 August 2013.
\item ACM WSDM: \href{http://www.wsdm-conference.org/}{Web Search and Data Mining}

WSDM (pronounced ``wisdom") is one of the premier conferences covering research in the areas of search and data mining on the Web. It publishes original, high quality papers and presentations related to search and data mining on the Web and the Social Web, with an emphasis on practical but principled novel models of search, retrieval and data mining, algorithm design and analysis, economic implications, and in-depth experimental analysis of accuracy and performance.

\item ICML: \href{http://icml.cc/2013/}{International Conference on Machine Learning}

The 30th International Conference on Machine Learning (ICML 2013) will be held in Atlanta, USA, on June 16 – June 21, 2013.

\item NIPS: \href{http://books.nips.cc/}{Neural Information Processing Systems Conference}

Papers may be only up to 8 pages long, including figures. An additional ninth page containing only cited references is allowed. Papers that exceed 9 pages will be rejected without review.

\item CIKM: \href{http://www.cikmconference.org/}{Conference on Information and Knowledge Management}

CIKM is a well-known top tier and premier ACM conference in the areas of information retrieval, knowledge management and databases. Since 1992, it has successfully brought together leading researchers and developers from the three communities. The purpose of the conference is to identify challenging problems facing the development of future knowledge and information systems, and to shape future research directions through the publication of high quality, applied and theoretical research findings.

The 22nd ACM International Conference on Information and Knowledge Management (\href{http://www.cikm2013.org/}{CIKM 2013}) will be held from October 27 to November 1, 2013 at San Francisco Airport Marriott Waterfront, Burlingame, CA, USA.

\item WWW: World-Wide Web Conference

The International World Wide Web Conference (abbreviated as WWW) is a yearly international academic conference on the topic of the future direction of the World Wide Web. It began in 1994 and is organized by the International World Wide Web Conferences Steering Committee (IW3C2). It is aimed at "key influencers, decision makers, technologists, businesses and standards bodies". The event usually spreads over 5 days.

The conference series is aimed at providing a global forum for discussion and debate in regards to the standardization of its associated technologies and the impact of said technologies on society and culture. Developers, researchers, users, and commercial ventures are all brought together by the conference to discuss the evolution of the Web. The conferences are organized by the IW3C2 in collaboration with Local Organizing Committees and Technical Program Committees

The 22nd International World Wide Web Conference (\href{http://www2013.org/}{WWW 2013}) will be held on 13th-17th, May in Rio de Janeiro Brazil.
\item IJCAI: \href{http://ijcai.org/}{International Joint Conference on Artificial Intelligence}

IJCAI is the International Joint Conference on Artificial Intelligence, the main international gathering of researchers in AI. Held biennially in \textbf{odd-numbered years} since 1969, IJCAI is sponsored jointly by IJCAI and the national AI societie(s) of the host nation(s).

\item AAAI: \href{http://www.aaai.org/Conferences/AAAI/aaai.php}{American Association for Artificial Intelligence}

IJCAI和AAAI轮流召开,论文页数较短(6页)

\item KDD: ACM SIGKDD Conference on Knowledge Discovery and Data Mining

\item ICPR: \href{http://www.icpr2014.org/tracks.htm}{International Conference on Pattern Recognition}
\end{enumerate}

\section{国际期刊}
\begin{enumerate}[(1)]
\item IR: \href{http://link.springer.com/journal/10791}{Information Retrieval}

The Journal of Information Retrieval is an international forum for theory, algorithms, and experiments that concern search and storage of text, images, video, and other such data. Research results published in the journal typically address the problems that arise for user-oriented tasks where the meaning as well as the explicit content of the data is of interest.

Information Retrieval features theoretical, experimental and applied papers. Theoretical papers report a significant conceptual advance in the design of algorithms or other processes for some information retrieval task. Experimental papers detail a test of one or more theoretical ideas in a laboratory or natural setting. Application papers cover successful application of some already established technique to a significant real world problem involving information retrieval.

Information retrieval overlaps with a variety of technical and behavioral fields. As a result, the journal includes papers which unify concepts across several traditional disciplinary boundaries, with specific application to problems of information retrieval.

\item JMLR: \href{http://jmlr.csail.mit.edu/}{Journal of Machine Learning Research}

JMLR has a commitment to rigorous yet rapid reviewing. Final versions are published electronically (ISSN 1533-7928) immediately upon receipt.

\textbf{JMLR Scope}: new algorithms with empirical, theoretical, psychological, or biological justification; experimental and/or theoretical studies yielding new insight into the design and behavior of learning in intelligent systems; accounts of applications of existing techniques that shed light on the strengths and weaknesses of the methods; formalization of new learning tasks (e.g., in the context of new applications) and of methods for assessing performance on those tasks; development of new analytical frameworks that advance theoretical studies of practical learning methods; computational models of data from natural learning systems at the behavioral or neural level; or extremely well-written surveys of existing work.

\item KNOWL-BASED SYST: \href{http://www.journals.elsevier.com/knowledge-based-systems/}{Knowledge-Based Systems}

Knowledge-Based Systems is the international, interdisciplinary and applications-oriented journal on KBS.

Knowledge-Based Systems focuses on systems that use knowledge-based techniques to support human decision-making, learning and action. Such systems are capable of cooperating with human users and so the quality of support given and the manner of its presentation are important issues. The emphasis of the journal is on the practical significance of such systems in modern computer development and usage.

As well as being concerned with the implementation of knowledge-based systems, the journal covers the design process, the matching of requirements and needs to delivered systems and the organisational implications of introducing such technology into the workplace and public life, expert systems, application of knowledge-based methods, integration with conventional technologies, software tools for KBS construction, decision-support mechanisms, user interactions, organisational issues, knowledge acquisition, knowledge representation, languages and programming environments, knowledge-based implementation techniques and system architectures. Also included are publication reviews.

Index bound in last issue of calendar year.

Benefits to authors
We also provide many author benefits, such as free PDFs, a liberal copyright policy, special discounts on Elsevier publications and much more. Please click here for more information on our author services.

Please see our Guide for Authors for information on article submission. If you require any further information or help, please visit our support pages: http://support.elsevier.com

Knowledge-Based Systems是SCI检索期刊,目前影响因子是4.104,投稿周期6个月。

\item Journal of Computational Information Systems

\href{http://www.jofcis.com/index.aspx}{Journal of Computational Information Systems (JCIS)} is an international forum for scientists and engineers in all aspects of computer science and technology publishing high quality and refereed papers. JCIS published by Binary Information Press, is indexed by \textbf{EI} Compendex. The papers for publication in JCIS are selected through rigorous peer reviews to ensure originality, timeliness, relevance, and readability. JCIS also seeks clearly written survey and review articles from experts in the field to promote insightful understanding of the state-of-the-art and the technology trends.

Published now monthly, JCIS aims to provide a high-level international forum for scientists and researchers to present the state of the art of Computer Science Theory, Software Engineering and Tools, Computer Networks and Communications, Computer hardware architecture, Information Security, Artificial Intelligence, Multimedia Technology and Application, Parallel processing, Computer Graphics, Image Processing, Chinese information processing, CAD/CAM, Information Systems, Database and Knowledge Discovery, Computer Applications and Other topics.

The current acceptance rate of this journal is below 20\%. A large number of submissions were rejected because of either poor quality or their subject areas were out of the scope and coverage of the publication.

Paper publication Fee: HK\$3600. Carefully observe the length limit of 8 pages for each paper. Extra pages beyond 8 pages are charged at HK\$300 per page.

JCIS示例文章:\href{http://www.jofcis.com/publishedpapers/2013\_9\_13\_5387\_5394.pdf}{Learning to rank based query expansion for patent retrieval}。
\end{enumerate}

\section{国内期刊}
\begin{enumerate}
\item 计算机应用(成都)

在线投稿,审稿周期三个月,发表周期半年。审稿费50元。

\item 计算机科学(重庆)

审稿周期两个月,发表周期8个月至一年左右,可以加急。无需审稿费。

\item 计算机工程(上海)

审稿周期两个月,发表周期一年左右,可以加急。审稿费50元。

\item 计算机应用研究(成都)

审稿周期三个月,发表周期15个月左右,可以加急。无需审稿费。
\end{enumerate}
